Unsere Welt wird von Tag zu Tag digitaler und vernetzter, dadurch wächst die Menge und Komplexität der Daten stätig an.
Eine Analyse dieser Daten birgt großes Potential für die Wissenschaft und Industrie.
Ein Teilgebiet dieser Daten ist in Form von hierarchischen Netzwerken strukturiert. Einige Anwendungsgebiete sind beispielsweise die medizinische Forschung, hier werden Verbindungen, Gruppen und Cluster Zugehörigkeiten von Krankheiten untersucht. In der sozialwissenschaftlichen Forschung kommen hierarchischen Netzwerke in Organigrammen zum Einsatz. Aber auch in Teilbereichen der Informatik, wie Build-, Dependency- und Source Code Versionierungs Management Software, finden sich hierarchische Verbindungen bei Abhängigkeiten von Software Modulen, Versionen und Schichten Software Architektur. 

Allerdings wird das strukturierte Analysieren dieser komplexen und großen Datenmengen mit klassischen zweidimensionalen Visualisierungen zunehmend schwieriger.
Da mit den exponentiell steigenden Datenmengen der letzten Jahre immer häufiger Überlappungen und andere visuelle Artefakte entstehen. 
Daher werden neue Methoden und Techniken benötigt, um den Analyse Prozess zu optimieren. In dieser Bachelorarbeit untersuchen wir einen neuen Ansatz um hierarchische Netzwerke zu visualisieren. Dabei erweitern wir bisherige zweidimensionale Konzepte von hierarchischen Netzwerken mit einer dritten Dimension und visualisieren das Ergebnis mittels eines Virtual Reality Systems. Virtual Reality bietet viele Vorteile wie beispielsweise einen verbesserten räumlichen Eindruck sowie viele intuitive Interaktionsmöglichkeiten mittels raumfüllenden VR Tracking Systemen. 
Wir glauben, dass es mit den Vorteilen von Virtual Reality gelingen kann Visualisierungen zu erstellen welche die dreidimensionale Informationsvisualisierung effizienter nutzen können. Dadurch soll es möglich sein größere und komplexere hierarchische Netzwerke besser als mit herkömmlichen zweidimensionalen Visualisierungen zu analysieren.