\chapter{Related Work}
Seitenanzahl anderer Arbeiten: Wilangyman: 3, BibTeX Consistency Tool: 3, Visual Comparison of Spatial Deviations via Geospatial Slicing 6 (viel Bilder), 

\section{2D and 3D Visualizations}

Although the focus of this thesis is on visualizing hierarchical networks in virtual reality, many concepts are the same or similar to its 2D or 3D counterparts. So first we want to give a brief overview of related 2D and 3D approaches in the field of network visualization.\\
Dynamic network visualization describes a research field where none or only few assumptions about the data are made, the technique usually works by force-based layouts using node-link diagrams as we already summarized in the background section \ref{exp:force_based_background}. Furthermore, we want to discuss approaches dealing with hierarchical and multilayer structured datasets.

\subsection{Hierarchical Visualizations}

As we read in \ref{exp:tree} before, a preferred data structure to encode hierarchical information are trees.
Schulz \cite{schulz_treevisnet_2011} presented a good overview of different tree visualization techniques. A rough separation can be made by the representation of edges witch is either explicit or implicit. 

\subsubsection{Explicit approaches}
On explicit visualizations links are directly drawn as we have already seen in \ref{fig:simple_tree}. This core technique is used in numerous application domains and has been brought present over multiple scientific fields. Based on that concept researchers have developed many extensions.
LensTree \cite{song_lenstree_2006} for example, exchanges the axes and allows collapsing certain parts of the tree to display information structures like computer file directories. 
Armando \cite{arce-orozco_radial_2017} uses a radial instead of parallel axes to show the hierarchical relation, the root node is placed in the center and child nodes extends radial to all directions along the radius.
Munzer \cite{munzner_h3_1997} uses the hyperbolic space to display a large tree, in comparison to the other visualizations this one also uses a three-dimensional space to represent trees.
Robertson \cite{robertson_cone_1991} also make use of the 3D space in his publications of the cone-tree and cam-tree, these display the child nodes as a circle below respectively beside their parents.  

\subsubsection{Implicit approaches - space filling}
In some use cases not only hierarchical information is important but rather encoding an additional attribute like the size of a node. A common technique for implicit approaches are space filling concepts like we have seen before \ref{fig:hierarchicalCirclePlot}.
Shneiderman \cite{shneiderman_tree_1992} presented this approach 1992 in the form of a tree map where he visualized the required disk space distribution of a file system. In addition to a common circle packing algorithm \ref{fig:hierarchicalCirclePlot}, Görtler \cite{gortler_bubble_2018} developed a bubble tree map technique which optimized the used space and allows encoding of additional properties via various shapes, draw strokes and colors. Sunburst charts are another common space filling technique to display node sizes in tree structures. 
As for 3D representations, Wang \cite{wang_visualization_2006} shows us a circular tree map extending to the z axis using cylinders. Instead of cylinders Balzer \cite{balzer_hierarchy_2004} uses nested hemispheres to visualize software structures. 
However, research is not limited to the layout of the visualization, as Itoh \cite{itoh_hierarchical_2004} shows us with their performance optimized technique for generating a rectangle packing tree map.

Even when we just scratched the surface we can already conclude that there are many approaches with various explicit and implicit layouts tree structure visualizations. Their focus lies on mapping hierarchical relationships. Each node stand on its own, maybe it has additional multivariate attributes for example node size, but there is no relationship between nodes of different parents as this would create a cycle and furthermore breaking the definition of a tree.

Jonker\\
Graph mapping: Multi-scale community visualization of massive graph data\\
=> Sicher das man das streichen soll? Wurde von Manuela vorgeschlagen als Vergleich zum Firework plot soweit ich mich erinnern kann

%Shneiderman \\
%Tree visualization with tree-maps: 2-d space-filling approach\\
%\\
%Wang\\
%Visualization of large hierarchical data by circle packing\\
%In this paper a novel approach is described for tree visualization using nested circles. (2D+3D)\\
%\\
%Görtler\\
%Bubble Treemaps for Uncertainty Visualization\\
%\\
%Itoh\\
%Hierarchical data visualization using a fast rectangle-packing algorithm\\
%\\
%Schulz\\
%Treevis.net: A Tree Visualization Reference
%\subsubsection{Explicit approaches}
%Shi\\
%Hierarchical Focus+Context Heterogeneous Network Visualization\\
%OnionGraph, aggregated based on node attributes or network topology,
%best of both worlds.\\
%\\
%Balzer\\
%Hierarchy Based 3D Visualization of Large Software Structures\\
%\\
%Munzner\\
%H3: laying out large directed graphs in 3D hyperbolic space\\
%\\
%Mansmann\\
%Exploring OLAP aggregates with hierarchical visualization techniques
%\subsection{Dynamic Node-Link Network Visualization}
%Würde eventuell zum Aufbau des Chapters passen. \\
%Hierarchical visualizations => network visualizations (node-link, force layout) => multilayer visualizations (als Kombination von beiden)\\
%Oder Network Visualization + Force based komplett streichen?

%%Kobourov\\
%Spring Embedders and Force Directed Graph Drawing Algorithms\\
%In this survey we consider several classical algorithms \dots for large and dynamic graphs
\subsection{Multilayer Visualization}
Ghonie McGee\\
The State of the Art in Multilayer Network Visualization\\
\\
De Domenico\\
MuxViz: a tool for multilayer analysis and visualization of networks\\
We demonstrate the ability of muxViz to analyse and interactively visualize multilayer data using empirical genetic, neuronal and transportation networks https://github.com/manlius/muxViz\\
\\


\subsection{Clustered Graph Visualizations}
Visualizations for data structures simlar to ours are rare. The most matching approach can considered as clustered graph visualisations.

Peter Eades and Qing-Wen Feng\\
Multilevel Visualization of Clustered Graphs*\\
im endeffekt das gleiche wie wir nur in 2D\\
\\
Balzer\\
Level-of-Detail Visualization of Clustered Graph Layouts\\
3D cluster ansatz\\
\\


\section{VR Visualizations}

\subsection{Layouts}

\subsubsection{Force based layout approaches}
In VR we see common 2D layout approaches like simple node-link diagrams adapted for the use in a 3D virtual scene. Often these layouts use a force based approach to calculate the node positions. 

Drogemuller\\
Examining virtual reality navigation techniques for 3D network visualisations\\
Uses a normal 3D node-link layout as a baseline.\\
\\
Sorger\\
Immersive Analytics of Large Dynamic Networks via Overview and Detail Navigation\\
3D force layout for node-link diagram\\
\\
Yang\\
Embodied Navigation in Immersive Abstract Data Visualization:
Is Overview+Detail or Zooming Better for 3D Scatterplot\\
uses 3D Scatterplot similar to node-link
\subsubsection{Constraint layout approaches}
Other concepts use fixed constrains to place their nodes and links.

Sorger\\
Fisheye Layout\\
Orig Paper https://vis.csh.ac.at/vrnetexplorer/\\
\\
Kwon\\
A Study of Layout, Rendering, and Interaction Methods for Immersive Graph Visualization\\
Spherical graph layouts, the viewer is placed at the center of the sphere, on which the graph is laid out.\\
\\
Halpin\\
Exploring Semantic Social Networks Using Virtual Reality\\
2D flat layout + use of extrusion
%Büschel\\
%Augmented Reality Graph Visualizations\\
%We present an exploration of the design space for edge styles and discuss the results of a user study comparing six different edge variants.\\
%\\
\\
\\
\\
\subsection{Navigation}
Content:
\begin{itemize}
    \item "Minimap"/Worlds-in-Miniature (WIM) in VR
    \item Scaling
    \item Room scale vs Table vs Seating
    \item Overview + Detail 
\end{itemize}
%In terms of navigation and interaction Yang et al. \cite{yang_embodied_2020} explored the possibility of zooming and rotation of an entire graph and Drogemuller \cite{drogemuller_examining_2020} compared different navigation concepts.\\
Papers: \\
Yang\\
Embodied Navigation in Immersive Abstract Data Visualization:
Is Overview+Detail or Zooming Better for 3D Scatterplot\\
\\
M Usoh\\
Walking > walking-in-place > flying, in virtual environments\\
\\
Zielasko\\
Remain seated: towards fully-immersive desktop VR\\
\\
Drogemuller\\
Examining virtual reality navigation techniques for 3D network visualisations\\
\\
Wolfgang Stuerzlinger\\
Simulated Reference Frame: A Cost-Effective Solution to Improve Spatial Orientation in VR\\
aka. Nintendo Wii Board Navigation\\
\\
Wolfgang Stuerzlinger\\
Evaluating Automatic Parameter Control Methods for Locomotion in Multiscale Virtual Environments\\
AUTOMATIC DISTANCE CONTROL teleporting\\

\subsection{Interaction}
Content: 
\begin{itemize}
    \item Edge Filtering
    \item Raycast/Laserpointer Selection
\end{itemize}

Drogemuller\\
VRige: Exploring Social Network Interactions in Immersive Virtual Environments\\
Raycast+Filter Cube\\
\\
Wolfgang Stürzlinger\\
Analyzing the Trade-off between Selection and Navigation in VR\\
\\
Yi-Jheng Huang\\
A gesture system for graph visualization in virtual reality environments\\
different hand gesture interactions\\
\\

%\subsection{Advantages of Visualization in VR}
%Kraus\\
%The Impact of Immersion on Cluster Identification Tasks\\
%quantitative user study to investigate the impact of immersion on cluster identification tasks in scatterplot visualizations\\
%\\
%Doug A. Bowman\\
%Virtual Reality: How Much Immersion Is Enough?\\
%\\
%M Cordeil\\
%Immersive Collaborative Analysis of Network Connectivity: CAVE-style or Head-Mounted Display?\\
%\\