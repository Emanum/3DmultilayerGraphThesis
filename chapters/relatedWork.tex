\chapter{Related Work}
Seitenanzahl anderer Arbeiten: Wilangyman: 3, BibTeX Consistency Tool: 3, Visual Comparison of Spatial Deviations via Geospatial Slicing 6 (viel Bilder), 

\section{2D and 3D Visualizations}

Although the focus of this thesis is on visualizing hierarchical networks in virtual reality, many concepts are the same or similar to its 2D or 3D counterparts. So first we want to give a brief overview of related 2D and 3D approaches in the field of hierarchical visualization.

\subsection{Hierarchical Tree Visualizations}

\subsubsection{Space filling approaches}
Shneiderman \\
Tree visualization with tree-maps: 2-d space-filling approach\\
\\
Wang\\
Visualization of large hierarchical data by circle packing\\
In this paper a novel approach is described for tree visualization using nested circles. (2D+3D)\\
\\
Görtler\\
Bubble Treemaps for Uncertainty Visualization\\
\\
Itoh\\
Hierarchical data visualization using a fast rectangle-packing algorithm\\
\\
Schulz\\
Treevis.net: A Tree Visualization Reference
\subsubsection{Hierarchical approaches}
%Shi\\
%Hierarchical Focus+Context Heterogeneous Network Visualization\\
%OnionGraph, aggregated based on node attributes or network topology,
%best of both worlds.\\
%\\
Jonker\\
Graph mapping: Multi-scale community visualization of massive graph data\\
=> Sicher das man das streichen soll? Wurde von Manuela vorgeschlagen als Vergleich zum Firework plot soweit ich mich erinnern kann\\
\\
Balzer\\
Hierarchy Based 3D Visualization of Large Software Structures\\
\\
Munzner\\
H3: laying out large directed graphs in 3D hyperbolic space\\
\\
Mansmann\\
Exploring OLAP aggregates with hierarchical visualization techniques

\subsection{Network Visualization}
Würde eventuell zum Aufbau des Chapters passen. \\
Hierarchical visualizations => network visualizations (node-link, force layout) => multilayer visualizations (als Kombination von beiden)\\
Oder Network Visualization + Force based komplett streichen?

Kobourov\\
Spring Embedders and Force Directed Graph Drawing Algorithms\\
In this survey we consider several classical algorithms \dots for large and dynamic graphs
\subsection{Multilayer Visualization}
Ghonie McGee\\
The State of the Art in Multilayer Network Visualization\\
\\
De Domenico\\
MuxViz: a tool for multilayer analysis and visualization of networks\\
We demonstrate the ability of muxViz to analyse and interactively visualize multilayer data using empirical genetic, neuronal and transportation networks https://github.com/manlius/muxViz\\
\\
Peter Eades and Qing-Wen Feng\\
Multilevel Visualization of Clustered Graphs*\\
im endeffekt das gleiche wie wir nur in 2D\\
\\

\section{VR Visualizations}

\subsubsection{VR Specialized Layouts}
Kwon\\
A Study of Layout, Rendering, and Interaction Methods for Immersive Graph Visualization\\

Sorger\\
Immersive Analytics of Large Dynamic Networks via Overview and Detail Navigation\\
Orig Paper https://vis.csh.ac.at/vrnetexplorer/\\
\\
Kwon\\
A Study of Layout, Rendering, and Interaction Methods for Immersive Graph Visualization\\
considerations of layout, rendering, and interaction methods for visualizing graphs in an  immersive environment user study to evaluate our techniques\\
Strat: The viewer is placed at the center of the sphere, on which the graph is laid out.\\
\\
Büschel\\
Augmented Reality Graph Visualizations\\
We present an exploration of the design space for edge styles and discuss the results of a user study comparing six different edge variants.\\
\\
Yang\\
Embodied Navigation in Immersive Abstract Data Visualization:
Is Overview+Detail or Zooming Better for 3D Scatterplot\\
\\
Halpin\\
Exploring Semantic Social Networks Using Virtual Reality\\
\\

In terms of navigation and interaction Yang et al. \cite{yang_embodied_2020} explored the possibility of zooming and rotation of an entire graph and Drogemuller \cite{drogemuller_examining_2020} compared different navigation concepts.

\subsection{Advantages of Visualization in VR}
Kraus\\
The Impact of Immersion on Cluster Identification Tasks\\
quantitative user study to investigate the impact of immersion on cluster identification tasks in scatterplot visualizations\\
\\
Doug A. Bowman\\
Virtual Reality: How Much Immersion Is Enough?\\
\\
M Cordeil\\
Immersive Collaborative Analysis of Network Connectivity: CAVE-style or Head-Mounted Display?\\
\\

\subsection{Navigation}
Content:
\begin{itemize}
    \item "Minimap"/Worlds-in-Miniature (WIM) in VR
    \item Scaling
    \item Room scale vs Table vs Seating
    \item Overview + Detail 
\end{itemize}

Papers: \\

M Usoh\\
Walking > walking-in-place > flying, in virtual environments\\
\\
Zielasko\\
Remain seated: towards fully-immersive desktop VR\\
\\
Drogemuller\\
Examining virtual reality navigation techniques for 3D network visualisations\\
\\
Wolfgang Stuerzlinger\\
Simulated Reference Frame: A Cost-Effective Solution to Improve Spatial Orientation in VR\\
aka. Nintendo Wii Board Navigation\\
\\

\textbf{Wolfgang Stuerzlinger\\
Evaluating Automatic Parameter Control Methods for
Locomotion in Multiscale Virtual Environments\\}


\subsection{Interaction}
Content: 
\begin{itemize}
    \item Edge Filtering
    \item Raycast/Laserpointer Selection
\end{itemize}

Yi-Jheng Huang\\
A gesture system for graph visualization in virtual reality environments\\
\\
Drogemuller\\
VRige: Exploring Social Network Interactions in Immersive Virtual Environments\\
Raycast+Filter Cube\\
\\
Wolfgang Stürzlinger\\
Analyzing the Trade-off between Selection and Navigation in VR\\



\section{Application and Libraries}
References: https://neo4j.com/developer/tools-graph-visualization/

\subsection{Applications}
Software\\
Gephi\\
\\
Software\\
Neo4j Bloom\\
\subsection{Libraries}
Bostock\\
D3: Data-Driven Documents\\
Examples\\
\\
Software\\
http://www.popotojs.com/ \\
\\
Software\\
https://visjs.org/