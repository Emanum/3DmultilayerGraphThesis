\chapter{Related Work}

Hint:\\
1.line: One Author \\
2.line: Title \\
3.line: Desc\\

\section{Visualization Techniques}

Shneiderman\\
The Eyes Have It: A Task by Data Type Taxonomy for Information Visualizations\\
Visual Information Seeking Mantra\\
\\
Brath \\
3D InfoVis is Here to Stay: Deal with It\\
Warum allgemein 3D Infos Vis Vorteile hat\\
\\
Bostock\\
D3: Data-Driven Documents\\
Examples, wahrscheinlich wenig relevant\\

\section{Graph Visualizations}
Jonker\\
Graph mapping: Multi-scale community visualization of massive graph data\\
\\
Lee\\
Task Taxonomy for Graph Visualization\\
a list of tasks for graph visualization that has
enough detail and specificity to be useful to: 1) designers who
want to improve their system and 2) to evaluators who want to
compare graph visualization systems.\\
\\
Software\\
Gephi\\

\section{(2D) Multilayer Visualizations}

Ghonie McGee\\
The State of the Art in Multilayer Network Visualization\\
\\
Shi\\
Hierarchical Focus+Context Heterogeneous Network Visualization\\
OnionGraph, aggregated based on node attributes or network topology,
best of both worlds.\\


\section{Force Graph}
Kobourov\\
Spring Embedders and Force Directed Graph Drawing Algorithms\\
Mehrere Algorithmen für Force Graph, Sehr detailliert und technisch \\
\\
Jacom\\
ForceAtlas2, a Continuous Graph Layout Algorithm for Handy Network Visualization Designed for the Gephi Software\\
Forcetlas2 is a force-directed layout close to other algorithms used for network spatialization. Integrate different techniques.\\
\\
Yifan Hu\\
Efficient, High-Quality Force-Directed Graph Drawing\\
Algorithmus für force Graph,  Erklärkung barnes hut etc Sehr detailliert, viel info\\

\section{VR Visualizations}

Sorger\\
Immersive Analytics of Large Dynamic Networks via Overview and Detail Navigation\\
Orig Paper https://vis.csh.ac.at/vrnetexplorer/\\
\\
Kwon\\
A Study of Layout, Rendering, and Interaction Methods for Immersive Graph Visualization\\
considerations of layout, rendering, and interaction methods for visualizing graphs in an  immersive environment user study to evaluate our techniques\\
Strat: The viewer is placed at the center of the sphere, on which the graph is laid out.\\
\\
Büschel\\
Augmented Reality Graph Visualizations\\
We present an exploration of the design space for edge styles and discuss the results of a user study comparing six different edge variants.\\

\section{Edge Bundeling - weglassen da nicht in eigener Arbeit verwendet}