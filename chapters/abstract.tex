Our world is getting more digital each year, new parts of our daily life getting connected and the amount and complexity of the produced data increases steadily.
The analysis of this data enables big opportunities for science and companies.
A subset of this data is organized in the form of hierarchical networks. We see this in multiple application domains for example medical research where connections of diseases are tracked, social science with different kind of relationships between people and software engineering with all kind of dependencies and version management connections. 

However, getting insight in this kind of data with traditional two-dimensional visualization is getting more difficult and therefore we need new methods and techniques to simplify and expedite the analysis process.
In this thesis we investigate a new approach to visualize hierarchical network data by extending already existing concepts of two-dimensional multilayer visualizations with a third dimension and applying it to a virtual reality based visualization system. We believe that the capabilities of virtual reality devices like improved spatial impression and interaction possibilities by room-scale tracked headsets and controllers allows the visualization to fully utilize the benefits of three-dimensional info visualization.