Our world is becoming more digital each year, new parts of our daily life getting connected and the amount and complexity of the produced data increases steadily.
The analysis of this data enables big opportunities for science and industry.
A subset of this data is organized in the form of hierarchical networks. We see this in multiple application domains for example medical research where connections, group and cluster memberships of diseases are tracked; Within the research area of social science with all kind of relationships between people like company organization charts; In build-, dependency- and source code version management software with their hierarchical connections as well and software engineering in general.

However, getting insight into this complex data with traditional two-dimensional visualization is getting more difficult as the visual clutter increases significantly with the exponentially growth of data we saw in recent years. Therefore, we need new methods and techniques to facilitate and expedite the analysis process.
In this thesis, we investigate a new approach to visualize hierarchical network data by extending already existing concepts of two-dimensional multilayer visualizations with a third dimension and applying it to a virtual reality based visualization system. We believe that the capabilities of virtual reality devices, such as improved spatial impression and interaction possibilities by room-scale tracked headsets and controllers allows the visualization to fully utilize the benefits of three-dimensional information visualization.