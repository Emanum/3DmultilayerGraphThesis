\chapter{Background}
Hint:\\
1.line: One Author \\
2.line: Title \\
3.line: Desc\\
\section{Visualization Techniques}
Papers:\\

Shneiderman\\
The Eyes Have It: A Task by Data Type Taxonomy for Information Visualizations\\
Visual Information Seeking Mantra\\
\\
Kobourov\\
Gestalt Principles in Graph Drawing\\
\\
Brath \\
3D InfoVis is Here to Stay: Deal with It\\
Warum allgemein 3D Infos Vis Vorteile hat\\
\\
Lee\\
Task Taxonomy for Graph Visualization\\
a list of tasks for graph visualization that has
enough detail and specificity to be useful to: 1) designers who
want to improve their system and 2) to evaluators who want to
compare graph visualization systems.\\
\\

\section{Network Visualization}

\subsection{Network Visualization Basics}

Kerren\\
\textbf{Introduction to Multivariate Network Visualization vorallem Chapter9: Heterogeneous Networks on Multiple Levels} \\
gute Zusammenfassung von Multilayer Network Visualization\\

West\\
Introduction to graph theory\\
\\

\subsection{Force-directed graph drawing}
alt:\\
Fruchterman\\
Graph drawing by force-directed placement\\
\\
Kamada Kawai\\
AN ALGORITHM FOR DRAWING GENERAL UNDIRECTED GRAPHS\\
\\

aktueller:\\
Yifan Hu\\
Efficient, High-Quality Force-Directed Graph Drawing\\
Algorithmus für force Graph,  Erklärkung barnes hut etc Sehr detailliert, viel info\\
\\
Kobourov\\
Spring Embedders and Force Directed Graph Drawing Algorithms\\
Mehrere Algorithmen für Force Graph, Sehr detailliert\\
\\

\section{VR Technology}

Content:
\begin{itemize}
    \item Describe the technologie stack (OpenVR, WEBXR/WEBVR, A-Frame, ThreeJS)
    \item Devices, HTC-VIVE, Oculus-Rift, Vendors
    \item Room scale vs Table vs Standing, possibilities of tracking
    \item 6 DOF vs 3 DOF (degrees of freedom)
    \item Vergleich HMD zu früheren Möglichkeiten mit "Cave" Virtual Reality
\end{itemize}

official Specs / Docs: 
\\
Desktop API:\\
https://www.khronos.org/openxr/ \\
https://github.com/ValveSoftware/openvr \\
\\
Web API:\\
   WebXR\\
https://immersiveweb.dev/ \\
https://github.com/immersive-web/webxr/blob/master/explainer.md \\
https://immersive-web.github.io/webxr/ \\
https://blog.mozvr.com/webxr-emulator-extension/ \\
\\
WebVR:(deprecated wird aber von unserer AFrame Version verwendet daher trotzdem relevant)\\
https://webvr.info/ \\
\\
Papers:\\
Cruz-Neira\\
The CAVE: audio visual experience automatic virtual environment\\
\\
M Cordeil\\
Immersive Collaborative Analysis of Network Connectivity: CAVE-style or Head-Mounted Display?\\
\\