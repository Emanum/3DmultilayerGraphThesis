\chapter{Background}

\section{Visualization Techniques}

All visualization techniques share some basic knowledge one of them is the famous “Visual Information-Seeking Mantra” by Shneiderman \cite{shneiderman_eyes_1996}.
\begin{quotation}
    Overview first, zoom and filter, then details-on-demand
\end{quotation}

Brath \\
3D InfoVis is Here to Stay: Deal with It\\
Warum allgemein 3D Infos Vis Vorteile hat\\

\section{Graph Theory}

\subsection{Graph Drawing}

Lee\\
Task Taxonomy for Graph Visualization\\
a list of tasks for graph visualization that has
enough detail and specificity to be useful to: 1) designers who
want to improve their system and 2) to evaluators who want to
compare graph visualization systems.\\
\\

\subsection{Force Based Graph Layouts}

Yifan Hu\\
Efficient, High-Quality Force-Directed Graph Drawing\\
Algorithmus für force Graph,  Erklärkung barnes hut etc Sehr detailliert, viel info\\
\\
Kobourov\\
Spring Embedders and Force Directed Graph Drawing Algorithms\\
Mehrere Algorithmen für Force Graph, Sehr detailliert und technisch \\
\\

\subsection{Tree Visualization}

\section{VR Technology}

\subsection{Devices}
6 degrees of freedom, different vendors, possibilities of tracking 

\subsection{APIs}
WebXR, WebVR

take a look at current technologies (HTC VIVE, WEBXR/WEBVR) and frameworks(AFrame, THREEJS, D3-force(3d)) with its capabilities
